\documentclass[a4paper]{ctexart}    %ctexart, article
\usepackage[margin=1cm]{geometry}
\usepackage{graphicx} % 插图
\usepackage{subfigure} % 子图包
\usepackage{amsmath, mathrsfs, amsfonts}
\usepackage{amssymb} % 更多公式符号
\usepackage{physics} % 公式输入便捷命令
\usepackage{hyperref} % 超链接
\hypersetup{
    colorlinks=true,
    linkcolor=blue,
    filecolor=magenta,
    urlcolor=cyan
}
\usepackage[utf8]{inputenc}
\usepackage{pgfplots}
\usepackage{xcolor}
\usepackage{tcolorbox}
\usepackage{minted}
\usepackage{tikz}
\usepackage{tkz-euclide}
\usepackage{array}
\usepackage{booktabs} % 调整表格线与上下内容的间隔


\begin{document}
\begin{tcolorbox}[
         colback=red!5!white,
         colframe=teal,
         title=\textbf{Install Seaborn}
    ]
\begin{minted}[mathescape,
               linenos,
               numbersep=5pt,
               gobble=0,
               frame=lines,
               framesep=2mm]{bash}
pip3 install scipy
pip3 install seaborn
\end{minted}
\end{tcolorbox}

\newpage
\begin{tcolorbox}[
         colback=red!5!white,
         colframe=teal,
         title=\textbf{seaborn (堆叠)柱状图1/2}
    ]
\begin{minted}[mathescape,
               linenos,
               numbersep=5pt,
               gobble=0,
               frame=lines,
               framesep=2mm]{python}
import matplotlib.pyplot as plt
import seaborn as sns
import pandas as pd
import numpy as np

# 设置风格样式: darkgrid, whitegrid, dark, white, ticks
sns.set_style('darkgrid')
# 解决中文乱码
plt.rcParams['font.sans-serif'] = ['SimHei']

# ------------------------------------
#              数据处理
# ------------------------------------
# 读取数据
df = pd.read_excel('xx.xlsx')
# 根据类别分组,计算每组中买家实际支付金额的总和
df1 = df.groupby['类别']['买家实际支付金额'].sum()
# 将消费总金额转成列表
num = np.array(list[df1])

# 根据类别和性别分组,统计不同买家的人数,并重置索引
df2 = df.groupby(['类别', '性别'])['买家会员名'].count().reset_index()
men_df = df2['性别'=='男']
women_df = df2['性别'=='女']

# 将男性和女性买家数转成列表
men_list = list(men_df['买家会员名'])
women_list = list(women_df['买家会员名'])

# 计算男性用户比例
ratio = np.array(men_list) / (np.array(men_list) + np.array(women_list))

# 设置输出的精度
np.set_printoptions(precision=2)

# 计算男女性的消费金额
men = num*ratio
women = num*(1-ratio)

# 删除重复数据
df3 = df2.drop_duplicates(['类别'])

# 将类别转成列表
name = list(df3['类别'])
\end{minted}
\end{tcolorbox}

\newpage
\begin{tcolorbox}[
         colback=red!5!white,
         colframe=teal,
         title=\textbf{seaborn (堆叠)柱状图2/2}
    ]
\begin{minted}[mathescape,
               linenos,
               numbersep=5pt,
               gobble=0,
               frame=lines,
               framesep=2mm]{python}
# ---------------------------------------------------
#                    开始绘制图表
# ---------------------------------------------------
width = 0.5
x = np.arange(len(name))
# 男性柱状图
plt.bar(x, men, width=width, color='slateblue', label='男性用户')
# 女性柱状图
plt.bar(x, women, width=width, color='orange', label='女性用户', bottom=men)

# 设置x轴与y轴的标签
plt.xlabel('消费类别')
plt.ylabel('男女分布')
plt.xticks(x, name, rotation=20)

# 在图表上显示数字文本
for a,b in zip(x,men):
    plt.text(a,b, '%.0f'%b, ha='center', va='top', fontsize=12)
for a,b,c in zip(x,women,men):
    plt.text(a,b+c, '%.0f'%b, ha='center', va='bottom', fontsize=12)

# 添加图例
plt.legend()
# 显示图表
plt.show()
\end{minted}
\end{tcolorbox}

\end{document}
