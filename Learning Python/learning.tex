\documentclass[a4paper]{ctexart}    %ctexart, article
\usepackage[margin=1cm]{geometry}
\usepackage{graphicx} % 插图
\usepackage{subfigure} % 子图包
\usepackage{amsmath, mathrsfs, amsfonts}
\usepackage{amssymb} % 更多公式符号
\usepackage{physics} % 公式输入便捷命令
\usepackage{hyperref} % 超链接
\hypersetup{
    colorlinks=true,
    linkcolor=blue,
    filecolor=magenta,
    urlcolor=cyan
}
\usepackage[utf8]{inputenc}
\usepackage{pgfplots}
\usepackage{xcolor}
\usepackage{tcolorbox}
\usepackage{minted}
\usepackage{tikz}
\usepackage{tkz-euclide}
\usepackage{array}
\usepackage{booktabs} % 调整表格线与上下内容的间隔


\begin{document}
\begin{tcolorbox}[
         colback=red!5!white,
         colframe=teal,
         title=\textbf{字符串}
    ]
\begin{minted}[mathescape,
               linenos,
               numbersep=5pt,
               gobble=0,
               frame=lines,
               framesep=2mm]{python}
# 1.字符串格式化--千分位分隔语法
'{:,.2f}'.format(1314.256)

# 2.将列表转化为字符串
str1 = ''.join(list1)

# 3.更改字符串
str1 = str1.replace('mm', 'xx')
\end{minted}
\end{tcolorbox}

\newpage
\begin{tcolorbox}[
         colback=red!5!white,
         colframe=teal,
         title=\textbf{zip}
    ]
\begin{minted}[mathescape,
               linenos,
               numbersep=5pt,
               gobble=0,
               frame=lines,
               framesep=2mm]{python}
# 内置的zip函数允许使用for循环并行访问多个序列;
# zip的输入参数是一个或多个序列,它的返回值是将这些序列并排的元素配对得到元组的列表;
# 在Python3中zip返回的是一个可迭代对象,要想显示结果需要将其包含在list调用中。

l1 = [1, 2, 3, 4]
l1 = [5, 6, 7, 8]
zip(l1, l2)
-> <zip object at 0x02623C8>
list(zip(l1,l2))
-> [(1,5),(2,6),(3,7),(4,8)]

for (x,y) in zip(l1,l2):
    print(x,y,'--',x+y)

# 使用zip构造字典 ---- for循环
keys = ['spam','eggs','toast']
vals = [1,3,5]
D1={}
for (k,v) in zip(keys, vals):
    D1[k] = v

# 使用zip构造字典 ---- 字典推导
{k: v for (k, v) in zip(keys, vals)}
\end{minted}
\end{tcolorbox}


\end{document}
